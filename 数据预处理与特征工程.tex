\documentclass[a4paper, 11pt]{article}

%%%%%% 导入包 %%%%%%
\usepackage{CJK}
\usepackage{graphicx}
\usepackage[unicode]{hyperref}
\usepackage{xcolor}
\usepackage{amsmath}
\usepackage{cite}
\usepackage{indentfirst}
\usepackage{enumitem}
%%%%%% 设置字号 %%%%%%
\newcommand{\chuhao}{\fontsize{42pt}{\baselineskip}\selectfont}
\newcommand{\xiaochuhao}{\fontsize{36pt}{\baselineskip}\selectfont}
\newcommand{\yihao}{\fontsize{28pt}{\baselineskip}\selectfont}
\newcommand{\erhao}{\fontsize{21pt}{\baselineskip}\selectfont}
\newcommand{\xiaoerhao}{\fontsize{18pt}{\baselineskip}\selectfont}
\newcommand{\sanhao}{\fontsize{15.75pt}{\baselineskip}\selectfont}
\newcommand{\sihao}{\fontsize{14pt}{\baselineskip}\selectfont}
\newcommand{\xiaosihao}{\fontsize{12pt}{\baselineskip}\selectfont}
\newcommand{\wuhao}{\fontsize{10.5pt}{\baselineskip}\selectfont}
\newcommand{\xiaowuhao}{\fontsize{9pt}{\baselineskip}\selectfont}
\newcommand{\liuhao}{\fontsize{7.875pt}{\baselineskip}\selectfont}
\newcommand{\qihao}{\fontsize{5.25pt}{\baselineskip}\selectfont}

%%%% 设置 section 属性 %%%%
\makeatletter
\renewcommand\section{\@startsection{section}{1}{\z@}%
{-1.5ex \@plus -.5ex \@minus -.2ex}%
{.5ex \@plus .1ex}%
{\normalfont\sihao\CJKfamily{hei}}}
\makeatother

%%%% 设置 subsection 属性 %%%%
\makeatletter
\renewcommand\subsection{\@startsection{subsection}{1}{\z@}%
{-1.25ex \@plus -.5ex \@minus -.2ex}%
{.4ex \@plus .1ex}%
{\normalfont\xiaosihao\CJKfamily{hei}}}
\makeatother

%%%% 设置 subsubsection 属性 %%%%
\makeatletter
\renewcommand\subsubsection{\@startsection{subsubsection}{1}{\z@}%
{-1ex \@plus -.5ex \@minus -.2ex}%
{.3ex \@plus .1ex}%
{\normalfont\xiaosihao\CJKfamily{hei}}}
\makeatother

%%%% 段落首行缩进两个字 %%%%
\makeatletter
\let\@afterindentfalse\@afterindenttrue
\@afterindenttrue
\makeatother
\setlength{\parindent}{2em}  %中文缩进两个汉字位


%%%% 下面的命令重定义页面边距,使其符合中文刊物习惯 %%%%
\addtolength{\topmargin}{-54pt}
\setlength{\oddsidemargin}{0.63cm}  % 3.17cm - 1 inch
\setlength{\evensidemargin}{\oddsidemargin}
\setlength{\textwidth}{14.66cm}
\setlength{\textheight}{24.00cm}    % 24.62

%%%% 下面的命令设置行间距与段落间距 %%%%
\linespread{1.4}
% \setlength{\parskip}{1ex}
\setlength{\parskip}{0.5\baselineskip}
\hypersetup{CJKbookmarks=true}
%%%% 正文开始 %%%%
\begin{document}
\begin{CJK*}{GBK}{song}

%%%% 定理类环境的定义 %%%%
\newtheorem{example}{例}             % 整体编号
\newtheorem{algorithm}{算法}
\newtheorem{theorem}{定理}[section]  % 按 section 编号
\newtheorem{definition}{定义}
\newtheorem{axiom}{公理}
\newtheorem{property}{性质}
\newtheorem{proposition}{命题}
\newtheorem{lemma}{引理}
\newtheorem{corollary}{推论}
\newtheorem{remark}{注解}
\newtheorem{condition}{条件}
\newtheorem{conclusion}{结论}
\newtheorem{assumption}{假设}

%%%% 重定义 %%%%
\renewcommand{\contentsname}{目录}  % 将Contents改为目录
\renewcommand{\abstractname}{摘要}  % 将Abstract改为摘要
\renewcommand{\refname}{参考文献}   % 将References改为参考文献
\renewcommand{\indexname}{索引}
\renewcommand{\figurename}{图}
\renewcommand{\tablename}{表}
\renewcommand{\appendixname}{附录}
\renewcommand{\algorithm}{算法}


%%%% 定义标题格式,包括title,author,affiliation,email等 %%%%
\title{特征工程与数据预处理}
\date{\today}
\author{张翼鹏}


\maketitle
\tableofcontents

\newpage


\section{特征工程是什么}
“数据和特征决定了机器学习的上限,而模型和算法只是逼近这个上限而已。”
\\ \indent
特征工程指的是把原始数据转变为模型的训练数据的过程,目的是最大限度地从原始数据中提取特征以供算法和模型使用。
\section{数据预处理}
未经处理的特征可能有以下问题:
\begin{enumerate}
\renewcommand{\labelenumi}{$\bullet$}
\item {\bf 不属于同一量纲}:即特征的规格不一样,不能够放在一起比较。无量纲化可以解决这一问题。
\item {\bf 信息冗余}:对于某些定量特征,其包含的有效信息为区间划分,例如学习成绩,假若只关心“及格”或“不及格”,那么需要将定量的分数,转换成“1”和“0”表示及格和不及格。二值化可以解决这一问题。
\item {\bf 定性特征不能直接使用}:某些机器学习算法和模型只能接受定量特征的输入,那么需要将定性特征转换为定量特征。最简单的方式是为每一种定性值指定一个定量值,但是这种方式过于灵活,增加了调参的工作。通常使用哑编码的方式将定性特征转换为定量特征。
\item {\bf 存在缺失值}:缺失值需要补充。
\end{enumerate}
\subsection{无量纲化}
\subsubsection{标准化}
$$x' = {{x - \mu } \over \sigma }$$

令每一{\bf 列}数据符合标准正态分布。
\subsubsection{归一化}
$$x' = {{x - \min } \over {\max  - \min }}$$

将每一{\bf 列}数据缩放到$[0,1]$区间内。
\subsubsection{正则化}
$$x' = {x \over {{{\left\| x \right\|}_p}}}$$

将每一{\bf 行}数据缩放到单位范数。(该步骤只在少数问题中使用,比如后面需要使用点积或其它核方法计算两个样本之间的相似性时。主要应用于文本分类和聚类中。)
\subsection{定量特征二值化(离散化)}
设定一个阈值,大于阈值的赋值为1,小于等于阈值的赋值为0:
$$ x'=\left\{
\begin{aligned}
&1 \quad x>{\rm{threshold}}\\
&0 \quad x \le {\rm{threshold}}\\
\end{aligned}
\right.
$$

进一步,我们用此法同样可以将数据离散化,比如讨论年龄段问题时将$[0,100]$区间离散化,均匀分成10个互不相交的小区间,分别赋值$0-9$。
\subsection{定性特征哑编码}
假设有$N$种定性值,则将这一个特征扩展为$N$种特征,即升维。当原始特征值为第$i$种定性值时,第$i$个扩展特征赋值为1,其他扩展特征赋值为0。

哑编码的方式相比直接指定的方式,不同的特征值之间没有序关系且互不影响,不用增加调参的工作。对于线性模型来说,使用哑编码后的特征可以达到非线性的效果。
\section{特征选择}
特征选择的原因:
\begin{enumerate}
\renewcommand{\labelenumi}{$\bullet$}
\item 降低复杂度、降维,减少数据冗余,使模型泛化能力更强,避免过拟合
\item 增强对特征和特征值之间的理解。
\end{enumerate}

特征选择的依据:
\begin{enumerate}
\renewcommand{\labelenumi}{$\bullet$}
\item {\bf 特征是否发散}:如果一个特征不发散,例如方差接近于0,也即样本在这个特征上基本上没有差异,这个特征对于样本的区分并无作用。
\item {\bf 特征与目标的相关性}:与目标相关性高的特征,应当优选选择。
\item {\bf 特征与特征的相关性}:若两个不同的特征相关性很强,说明二者携带的信息有很大的重复,易造成信息冗余,此时需要舍弃一个特征,或利用二者构造一个新的特征。
\end{enumerate}

根据特征选择的形式可以将特征选择方法分为3种:
\begin{enumerate}
\renewcommand{\labelenumi}{$\bullet$}
\item {\bf Filter(过滤法)}:按照发散性或者相关性对各个特征进行评分,根据评分选择特征。
\item {\bf Wrapper(包装法)}:根据目标函数(通常是预测效果评分),每次选择若干特征,或者排除若干特征。
\item {\bf Embedded(嵌入法)}:先使用某些机器学习的算法和模型进行训练,得到各个特征的权值系数,根据系数从大到小选择特征。类似于Filter方法,但是是通过训练来确定特征的优劣。
\end{enumerate}
\subsection{过滤法}
\subsubsection{移除低方差的特征}
设定阈值,将方差小于该阈值的特征剔除。
\subsubsection{单变量特征选择}
单独地计算每个特征的某个统计指标,根据该指标来判断哪些特征重要,剔除那些不重要的特征。
\begin{enumerate}
\renewcommand{\labelenumi}{$\bullet$}
\item 分类问题($y$离散)可采用:卡方检验,互信息。
\item 回归问题($y$连续)可采用:皮尔森相关系数,最大信息系数。
\end{enumerate}
1、卡方检验
$${\chi ^2} = \sum {{{{{(A - E)}^2}} \over E}} $$

\hangafter 1
\hangindent 5em
作用:检验定性自变量对定性因变量的相关性。$A$为观察值,$E$为期望值。检验统计量越大,说明两变量相关关系越强。

\noindent
2、皮尔森相关系数
$${\rho _{X,Y}} = {{{\mathop{\rm cov}} (X,Y)} \over {{\sigma _X}{\sigma _Y}}}$$

作用:衡量变量之间的线性相关性,结果的取值区间为$[-1,1]$。

\hangafter 1
\hangindent 5em
优势:计算速度快;取值区间是$[-1,1]$,使其能够表示更丰富的关系。

\hangafter 1
\hangindent 5em
缺陷:只对线性关系敏感,如果关系是非线性的,即便两个变量具有一一对应的关系,皮尔森相关性也可能会接近0。

\noindent
3、互信息
$$I(X;Y) = \sum\limits_{x \in X} {\sum\limits_{y \in Y} {p(x,y)\log {{p(x,y)} \over {p(x)p(y)}}} } $$

作用:评价定性自变量与定性因变量的相关性。

\hangafter 1
\hangindent 5em
缺陷:没有办法归一化,在不同数据集上的结果无法做比较;对于连续变量的计算不是很方便,通常变量需要先离散化,但互信息的结果对离散化的方式很敏感。

\noindent
4、最大互信息系数($MIC$)
$$MIC = \mathop {\max }\limits_{{\rm{m}}*n < B} {{\mathop {\max }\limits_{\text{不同划分}} (\sum\limits_{1 \le x \le m} {\sum\limits_{1 \le y \le n} {p(x,y)\log {{p(x,y)} \over {\sum\limits_{1 \le k \le m} {p(k,y)} \sum\limits_{1 \le k \le n} {p(x,k)} }}} } )} \over {\log \min \{ m,n\} }}$$

只讨论数据的两个属性(随机变量),数据点分布在二维空间中,用$m*n$的网格划分数据空间,将落在第$(x,y)$格子中的数据点的频率作为$P(x,y)$ 的估计,然后计算离散化后的随机变量的互信息。因为$m*n$的网格划分方式不止一种,所以我们要获得使互信息最大的网格划分,然后使用归一化因子,将互信息的值转化为$(0,1)$区间之内。最后,找到能使归一化互信息最大的网格分辨率(即$m$和$n$的值,$B=n^{0.6}$),作为$MIC$的度量值。

\hangafter 1
\hangindent 5em
优势:若具有足够的样本,可以捕获广泛的关系,而不限于特定的函数类型;对类型不同但噪声程度同等的关系给予相近的分数。

缺陷:当零假设不成立时,$MIC$的统计可能会受到影响。
\subsection{包装法}
\subsubsection{递归特征消除}
对模型进行多轮训练,每轮训练后,移除若干权重(即系数)低的特征,再基于新的特征集进行下一轮训练,直到特征数量达到要求。
\subsection{嵌入法}
有些机器学习方法本身就具有对特征进行打分的机制,或者很容易将其运用到特征选择任务中。嵌入法即用基于机器学习模型的方法来选择特征。
\subsubsection{基于$L1$或$L2$范数的特征选择}
对所有原始特征进行训练,给代价函数加入$L1$或$L2$惩罚项:
$$\lambda ||w |{|_1}\quad\quad\lambda ||w |{|_2}$$
使用$L1$范数可以快速让某些特征的权重收缩到0;使用$L2$范数可以集体减小多个特征的权重。
\section{降维}
特征选择完成后,就可以直接训练模型了。但是由于特征矩阵有时过大,导致计算量大,训练时间长,因此降低特征矩阵维度也是必不可少的。常见的降维方法除了以上提到的基于$L1$惩罚项的模型以外,另外还有主成分分析法($PCA$)和线性判别分析法($LDA$),其本质都是将原始的样本映射到维度更低的样本空间中。
\subsection{$PCA$}
如果要将原始$D$维数据投影到$M$维子空间当中,$PCA$的做法是计算原始数据的协方差矩阵$S$,求其前$M$大特征值对应的单位特征向量,然后将其作用到原始样本矩阵上,得到降维后的样本矩阵。

$PCA$的目标是去掉原始数据冗余的维度,让映射后的样本具有最大的发散性,是一种无监督的降维方法。
\subsection{$LDA$}
首先计算类间散度矩阵$S_b$:
$${S_b} = ({\mu _0} - {\mu _1}){({\mu _0} - {\mu _1})^T}$$
其中$\mu _0$是第0类样本的均值,$\mu _1$是第1类样本的均值。

然后计算类内散列矩阵$S_w$:
$${S_w} = \sum\limits_{x \in {X_0}} {(x - {\mu _0}){{(x - {\mu _1})}^T}}  + \sum\limits_{x \in {X_1}} {(x - {\mu _0}){{(x - {\mu _1})}^T}} $$
其中$X_0$是第0类样本的集合,$X_1$是第1类样本的集合。

最后${S_w}^{ - 1}{S_b}$的最大特征值所对应的特征向量$w$即为最佳投影方向。

$LDA$的目标是使相同类别的数据分布更紧凑,不同类别的数据尽量相互远离,让映射后的样本有最好的分类性能,是一种有监督的降维方法。
\end{CJK*}
\end{document}
